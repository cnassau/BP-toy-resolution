\documentclass[a4paper]{paper}
\usepackage{amsmath}
\usepackage{amsthm}
\usepackage{amsfonts}
\usepackage[alphabetic]{amsrefs}
\usepackage[T1]{fontenc}
\usepackage[utf8]{inputenc}
\usepackage{lmodern}
\usepackage[english]{babel}
\usepackage{sagetex}
\usepackage{csquotes}
\usepackage{tikz}
\usepackage{adjustbox}
\usepackage{inconsolata}

\newcommand{\id}{\operatorname{id}}
\newcommand{\BP}{BP}
\newcommand{\us}{_\ast}
\newcommand{\os}{^\ast}
\newcommand{\QQ}{\mathbb Q}
\newcommand{\FF}{\mathbb F}
\newcommand{\ZZ}{\mathbb Z}
\newcommand{\NN}{\mathbb N}

\newcommand{\stt}[1]{{\tt#1}}

\newcommand{\scrb}{{\mathcal B}}
\newcommand{\scrh}{{\mathcal H}}

\lstdefinestyle{SageInput}{
  style=DefaultSageInput,
  basicstyle={\small\linespread{0.8}\ttfamily},
}
\lstdefinestyle{SageOutput}{
  style=DefaultSageOutput,
  basicstyle={\small\linespread{0.8}\ttfamily},
}



\raggedright
\setlength{\parskip}{8pt plus 5pt minus 2pt}

\title{A toy implementation of the $\BP$-bar complex in Sage}
\author{Christian Nassau}
\date{\today}

\begin{document}

\maketitle

\begin{sagesilent}
load("bpbar.py")
\end{sagesilent}

We describe a small Sage library that implements enough of the Bar complex for
$\BP$ to compute some low-lying Novikov-Ext groups. More specifically,
we implement a class \stt{BPBar} that represents the infinite tensor power
\begin{align*}
    \scrb &= \pi\us\left(\BP\land\BP\land\cdots\right) = \Gamma\otimes_A\Gamma\otimes_A\Gamma\otimes_A\cdots
\intertext{and also a class \stt{HBPBar} which represents}
    \scrh &= H\us\left(\BP\land\BP\land\cdots\right) \subset \QQ\otimes {\mathcal B}
\end{align*}
Here $(A,\Gamma) = (\BP\us,\BP\us\BP)$ is the $\BP$-Hopf algebroid
and we think of $\BP^{\land\infty}$ as the colimit of the $\BP^{\land k}$
where $\BP^{\land k}\rightarrow \BP^{\land(k+1)}$ is given by $x\mapsto x\land 1$.

This code was an attempt to recreate some computations by Glen Wilson in Sage,
and we thank him heartily
for sharing his code and for many inspiring communications
during the Homotopy Theory Summer in Berlin in 2018.

\begin{section}{Overview of some classes}

The $\scrh$ and $\scrb$ come equipped with classes \stt{BasisEnumerator}, \stt{BarBasis}
that allow to loop through a basis of
a $\BP^{\land s}$ in a fixed dimension.
Specifically, an instance of \stt{BasisEnumerator} knows how to
\begin{enumerate}
\item loop through the weighted partitions $P_w(n)$ of an integer $n$ where $w=(w_1,\ldots,w_k)$ is an arbitrary
list of weights
\item efficiently compute the reverse map $P_w(n)\rightarrow \NN_0$ (function \stt{seqno})
\end{enumerate}
The \stt{BarBasis} translates this information from partitions to elements of $\scrb$ and $\scrh$.

The structure maps of $\scrb$ and $\scrh$ are all derived from the map $\eta_R$
which on $\BP^{\land\infty}$ is given by $x\mapsto 1\land x$. For the generators $m_k$, $t_j$ of $\scrh$
this is given by
$$\eta_R(m_n) = \sum_{i+j=n} m_i t_j^{p^i},\quad
\eta_R(\underbrace{1\vert\cdots\vert1\vert t_k}_{\text{$p$ ones}})
= \underbrace{1\vert\cdots\vert1\vert t_k}_{\text{$p+1$ ones}}.
$$
This is later used to compute the $\Delta(t_n)$ via $\Delta\eta_R(m_n) = (\eta_R\otimes\id)\eta_R(m_n)$.

In Sage one can compute these as follows (here for $p=3$):
\begin{sagecommandline}
sage: M = HBPBar(3) ; m,t = M.gens()
sage: e,D = M.etaR,M.Delta
sage: m[1], e(m[1])
sage: m[2], e(m[2])
sage: m[3], e(m[3])
sage: # an iterated eta_R of t2
sage: t[2], e(t[2]), e(e(t[2]))
sage: # two coproducts
sage: D(t[1])
sage: D(t[2])
\end{sagecommandline}

There is a class \stt{ArakiGens} that implements the isomorphisms
$\scrb\otimes\QQ \leftrightarrow \scrh\otimes\QQ$. This is used to transport the structure formulas
from \stt{HBPBar} to \stt{BPBar}:
\begin{sagecommandline}
sage: A = ArakiGens(3)
sage: A.mapm2v()
sage: A.mapm2v()(m[1])
sage: A.mapm2v()(m[2])
sage: A.mapv2m()
sage: A.mapv2m()(A.mapm2v()(m[2]))
\end{sagecommandline}

For $\scrb$ at $p=3$ we then get, for example
\begin{sagecommandline}
sage: B = BPBar(3) ; v,t = B.gens()
sage: B.etaR(v[1])
sage: B.etaR(v[2])
sage: B.Delta(t[1])
sage: B.Delta(t[2])
\end{sagecommandline}

One can use a different base ring for \stt{BPBar} to compute with a reduction
of $\scrb$ modulo some $p^N$:
\begin{sagecommandline}
sage: B = BPBar(3,IntegerModRing(3**2)) ; v,t = B.gens()
sage: B.Delta(t[3])
\end{sagecommandline}

For the bar complex we also need the \enquote{higher coproducts} $\Delta_k$
that are induced by the map
$$a_1\land a_2\land\cdots \mapsto
a_1\land\cdots a_k\land 1\land a_{k+1}\land\cdots$$
on spectra.
These are available as \stt{Delta\_map(k)}:
\begin{sagecommandline}
sage: X = BPBar(2) ; v,t = X.gens()
sage: e,D = X.etaR, X.Delta_map
sage: D(3)
sage: D(2)(t[1])
sage: D(2)(e(t[1]))
sage: D(2)(e(e(t[1])))
sage: D(2)(e(e(e(t[1]))))
\end{sagecommandline}

Under the hood a lot of (hopefully well-positioned) caching is going on, to make sure that
the basic structure formulas (including the $\eta_R(v_j)$ and $\Delta_k(t_j)$)
are only computed once.
This is particularly relevant when such a structure map is evaluated on a monomial:
\begin{sagecommandline}
sage: D(2)(e(t[1]**3*t[2]))
\end{sagecommandline}

\end{section}


\begin{section}{Cohomology}

    The \stt{BPBar} supports a method \stt{matrix\_differential} that computes the matrix
    of the bar differential $\delta:\Gamma^{\otimes s}\rightarrow\Gamma^{\otimes (s+1)}$,
    i.~e.~ the alternating sum of the $\Delta_k$.
    \begin{sageexample}
        sage: BPBar(3).matrix_differential(2,8)
    \end{sageexample}
    Here the matrix is computed with respect to the following bases:
    \begin{sageexample}
        sage: list(BPBar(3).bar_basis(2,8))
        sage: list(BPBar(3).bar_basis(3,8))
    \end{sageexample}
    The method \stt{Ext\_smith} goes one step further and computes the Smith normal form
    of the differential; for example, in bidegree (2,16) we find one diagonal coefficient that
    is divisible by $p$:
    \begin{sagecommandline}
    sage: d,u,v = BPBar(3).matrix_differential(2,16).smith_form(integral=True)
    sage: d.diagonal()
    \end{sagecommandline}
    This coefficient corresponds to an $x\in \Gamma^{\otimes 2}$ such that $\delta(x)\equiv 0\mod 6$;
    we therefore find a non-zero cocycle $y=\frac16\delta(x)$. The \stt{Ext\_smith} routine
    uses the transformation matrix \stt{u} to compute this cocycle:
    \begin{sagecommandline}
    sage: L = list(BPBar(3).Ext_smith(3,16,withcocycle=True))
    sage: # L is a list of tuples (cf,x,y) with delta(x) = cf*y
    sage: (c,x,y), = L
    \end{sagecommandline}
    Then
    $$c=\sage{c},\quad x=\sage{x}$$
    and
    $$y=\sage{y}.$$

    We conclude with some tables that illustrate the timing of such Ext computations and the
    size/growth of the bar resolution.

\end{section}
\begin{sagesilent}
def normpower(p,n):
    Zp = IntegerModRing(p)
    ans = 1
    while Zp(n) == 0:
        ans *= p
        n /= p
    return ans
def cachedcomputation(c):
    import pickle
    def cf(X,s,d):
        p=X.prime
        fname = "bpbar.cache.%d.%d.%d" % (p,s,d)
        if not os.path.exists(fname):
          with open(fname, 'wb') as f:
            pickle.dump(c(X,s,d),f)
            print("stored cached data in %s" % fname)
        with open(fname, 'rb') as f:
            print("loaded cached data from %s" % fname)
            L = pickle.load(f)
        return L
    return cf
@cachedcomputation
def extcomp(X,s,d):
    p = X.prime
    g=globals()
    cmd = "globals()['L']=list(BPBar(%d).Ext_smith(%d,%d))" % (X.prime,s,d)
    if True:
        secs = timeit(cmd,seconds=True,number=1)
        L = g['L']
    else:
        secs = 3.62123232
        L = [(p,1,1),(7*p*p,1,1)]
    return [ secs, "+".join(str(normpower(p,c)) for (c,x,y) in L)]
figcode="""\\begin{figure}[ht]
    \\begin{adjustbox}{addcode={\\begin{minipage}{0.8\\width}}{\\caption{%%
        %s
        }\\end{minipage}},rotate=270,center}
        \\begin{tikzpicture}[scale=1.4]
        %s\\end{tikzpicture}
    \\end{adjustbox}
  \\end{figure}"""
def getfig(X,smax,dmax):
   p = X.prime
   caption="Novikov Ext for the prime $%d$" % p
   dims = [d for d in range(1,smax+dmax+1) if d % (2*(p-1)) == 0 ]
   fcode = "\\draw (0,1) grid (%d,%d);\n" % (len(dims),smax+1)
   for s in range(1,smax+1):
      fcode += "\\node[anchor=east] at (-0.2,%f) {\\small $%d$};\n" % (s+0.5,s)
   for (i,d) in enumerate(dims):
      fcode += "\\node[anchor=center] at (%f,0.8) {\\small $%d$};\n" % (i+0.5,d)
   for s in range(1,smax+1):
      skip = False
      for (i,d) in enumerate(dims):
         if skip:
             secs, bas = "skipped", "?"
         else:
             secs, bas = extcomp(X,s,d)
             print("Computed (%d,%d) for p=%d in %s" % (s,d,p,secs))
             if secs>2000:
                 skip=True
             secs = "%.1fs" % secs
         fcode += "\\node[anchor=center] at (%f,%f) {%s};\n" % (i+0.5,s+0.7,"\\small %s" % bas)
         fcode += "\\node[anchor=center] at (%f,%f) {%s};\n" % (i+0.5,s+0.2,"\\scriptsize %s" % secs)
   return figcode % (caption,fcode)
\end{sagesilent}

\sagestr{getfig(BPBar(2),6,18)}

\sagestr{getfig(BPBar(3),6,40)}

\sagestr{getfig(BPBar(5),6,100)}


\begin{sagesilent}
def dimtable(X,maxs,ndims):
    p = X.prime
    dims = list(2*(p-1)*_ for _ in range(1,ndims))
    tab = "\\begin{figure}\\caption{Rank of $\\BP\\us\\BP^{\\land s}$ for the prime $"+str(X.prime)+"$}\n"
    tab += "\\medskip\\begin{center}\\begin{tabular}{c|r" + "r"*maxs + "}\n"
    tab += "dim&" + "&".join(str(_) for _ in range(maxs)) + "\\\\\n"
    tab += "\\hline\n"
    tab += "\\\\\n".join("%d&"%d +   "&".join(str(len(X.bar_basis(sdeg,d))) for sdeg in range(maxs)) for d in dims)
    tab += "\n\\end{tabular}\\end{center}"
    tab += "\n\\end{figure}"
    return tab
\end{sagesilent}

\sagestr{dimtable(BPBar(2),7,30)}
\sagestr{dimtable(BPBar(3),7,30)}
\sagestr{dimtable(BPBar(5),7,30)}


\end{document}
